\label{ch:01}
\SetRectoHeadText{Out to Sea}
\begin{ChapterStart}
\vspace*{2\nbs}

\ChapterTitle{CHAPTER\, ONE}
\vspace{1.5\nbs}
\ChapterSubtitle{Out to Sea}
\end{ChapterStart}

\noindent\charscale[2.0]{I} \textsc{had this story} from one who had no business to tell it to me, or to any other. I may credit the seductive influence of an old vintage upon the narrator for the beginning of it, and my own skeptical incredulity during the days that followed for the balance of the strange tale.

When my convivial host discovered that he had told me so much, and that I was prone to doubtfulness, his foolish pride assumed the task the old vintage had commenced, and so he unearthed written evidence in the form of musty manuscript, and dry official records of the British Colonial Office to support many of the salient features of his remarkable narrative.

I do not say the story is true, for I did not witness the happenings which it portrays, but the fact that in the telling of it to you I have taken fictitious names for the principal characters quite sufficiently evidences the sincerity of my own belief that it \emph{may} be true.

The yellow, mildewed pages of the diary of a man long dead, and the records of the Colonial Office dovetail perfectly with the narrative of my convivial host, and so I give you the story as I painstakingly pieced it out from these several various agencies.

If you do not find it credible you will at least be as one with me in acknowledging that it is unique, remarkable, and interesting.

From the records of the Colonial Office and from the dead man's diary we learn that a certain young English nobleman, whom we shall call John Clayton, Lord Greystoke, was commissioned to make a peculiarly delicate investigation of conditions in a British West Coast African Colony from whose simple native inhabitants another European power was known to be recruiting soldiers for its native army, which it used solely for the forcible collection of rubber and ivory from the savage tribes along the Congo and the Aruwimi.

The natives of the British Colony complained that many of their young men were enticed away through the medium of fair and glowing promises, but that few if any ever returned to their families.

The Englishmen in Africa went even further; saying that these poor blacks were held in virtual slavery, since when their terms of enlistment expired their ignorance was imposed upon by their white officers, and they were told that they had yet several years to serve.

And so the Colonial Office appointed John Clayton to a new post in British West Africa, but his confidential instructions centered on a thorough investigation of the unfair treatment of black British subjects by the officers of a friendly European power. Why he was sent, is, however, of little moment to this story, for he never made an investigation, nor, in fact, did he ever reach his destination.

Clayton was the type of Englishman that one likes best to associate with the noblest monuments of historic achievement upon a thousand victorious battle fields—a strong, virile man—mentally, morally, and physically.

In stature he was above the average height; his eyes were gray, his features regular and strong; his carriage that of perfect, robust health influenced by his years of army training. Political ambition had caused him to seek transference from the army to the Colonial Office and so we find him, still young, intrusted with a delicate and important commission in the service of the Queen.

When he received this appointment he was both elated and appalled. The preferment seemed to him in the nature of a well merited reward for painstaking and intelligent service, and as a stepping stone to posts of greater importance and responsibility; but, on the other hand, he had been married to the Hon.\ Alice Rutherford for scarce a three months, and it was the thought of taking this fair young girl into the dangers and isolation of tropical Africa that dismayed and appalled him.

For her sake he would have refused the appointment; but she would not have it so. Instead she insisted that he accept, and, indeed, take her with him.

There were mothers and brothers and sisters, and aunts and cousins to express various opinions on the subject, but as to what they severally advised history is silent.

We know only that on a bright May morning in 1888, John, Lord Greystoke, and Lady Alice sailed from Dover on their way to Africa.

A month later they arrived at Freetown where they chartered a small sailing vessel, the \emph{Fuwalda}, which was to bear them to their final destination.

And here John, Lord Greystoke, and Lady Alice, his wife, vanished from the eyes and from the knowledge of men.

Two months after they weighed anchor and cleared from the port of Freetown a half dozen British war vessels were scouring the south Atlantic for trace of them or their little vessel, and it was almost immediately that the wreckage was found upon the shores of St.~Helena which convinced the world that the \emph{Fuwalda} had gone down with all on board, and hence the search was stopped ere it had scarce begun; though hope lingered in longing hearts for many years.

The \emph{Fuwalda}, a barkantine of about one hundred tons, was a vessel of the type often seen in coastwise trade in the far southern Atlantic, their crews composed of the offscourings of the sea—unhanged murderers and cutthroats of every race and every nation.

The \emph{Fuwalda} was no exception to the rule. Her officers were swarthy bullies, hating and hated by their crew. The captain, while a competent seaman, was a brute in his treatment of his men. He knew, or at least he used, but two arguments in his dealings with them—a belaying pin and a revolver—nor is it likely that the motley aggregation he signed would have understood aught else.

So it was that from the second day out from Freetown John Clayton and his young wife witnessed scenes upon the deck of the \emph{Fuwalda} such as they had believed were never enacted outside the covers of printed stories of the sea.

It was on the morning of the second day that the first link was forged of what was destined to form a chain of circumstances ending in a life for one then unborn such as has probably never been paralleled in the history of man.

Two sailors were washing down the decks of the \emph{Fuwalda}, the first mate was on duty, and the captain had stopped to speak with John Clayton and Lady Alice.

The men were working backwards toward the little party who were facing away from the sailors. Closer and closer they came, until one of them was directly behind the captain. In another moment he would have passed by and this strange narrative had never been recorded.

But just that instant the officer turned to leave Lord and Lady Greystoke, and, as he did so, tripped against the sailor and sprawled headlong upon the deck, overturning the water-pail so that he was drenched in its dirty contents.

For an instant the scene was ludicrous; but only for an instant. With a volley of awful oaths, his face suffused with the scarlet of mortification and rage, the captain regained his feet, and with a terrific blow felled the sailor to the deck.

The man was small and rather old, so that the brutality of the act was thus accentuated. The other seaman, however, was neither old nor small—a huge bear of a man, with fierce black mustachios, and a great bull neck set between massive shoulders.

As he saw his mate go down he crouched, and, with a low snarl, sprang upon the captain crushing him to his knees with a single mighty blow.

From scarlet the officer s face went white, for this was mutiny; and mutiny he had met and subdued before in his brutal career. Without waiting to rise he whipped a revolver from his pocket, firing point blank at the great mountain of muscle towering before him; but, quick as he was, John Clayton was almost as quick, so that the bullet which was intended for the sailor’s heart lodged in the sailor’s leg instead, for Lord Greystoke had struck down the captain’s arm as he had seen the weapon flash in the sun.

Words passed between Clayton and the captain, the former making it plain that he was disgusted with the brutality displayed toward the crew, nor would he countenance anything further of the kind while he and Lady Greystoke remained passengers.

The captain was on the point of making an angry reply, but, thinking better of it, turned on his heel and black and scowling, strode aft.

He did not care to antagonize an English official, for the Queen’s mighty arm wielded a punitive instrument which he could appreciate, and which he feared—England’s far reaching navy.

The two sailors picked themselves up, the older man assisting his wounded comrade to rise. The big fellow, who was known among his mates as Black Michael, tried his leg gingerly, and, finding that it bore his weight, turned to Clayton with a word of gruff thanks.

Though the fellow’s tone was surly, his words were evidently well meant. Ere he had scarce finished his little speech he had turned and was limping off toward the forecastle with the very apparent intention of forestalling any further conversation.

They did not see him again for several days, nor did the captain vouchsafe them more than the surliest of grunts when he was forced to speak to them.

They messed in his cabin, as they had before the unfortunate occurrence; but the captain was careful to see that his duties never permitted him to eat at the same time.

The other officers were coarse, illiterate fellows, but little above the villainous crew they bullied, and were only too glad to avoid social intercourse with the polished English noble and his lady, so that the Claytons were left very much to themselves.

This in itself accorded perfectly with their desires, but it also rather isolated them from the life of the little ship so that they were unable to keep in touch with the daily happenings which were to culminate so soon in bloody tragedy.

There was in the whole atmosphere of the craft that undefinable something which presages disaster. Outwardly, to the knowledge of the Claytons, all went on as before upon the little vessel, but that there was an undertow leading them toward some unknown danger both felt, though they did not speak of it to each other.

On the second day after the wounding of Black Michael, Clayton came on deck just in time to see the limp body of one of the crew being carried below by four of his fellows while the first mate, a heavy belaying pin in his hand, stood glowering at the little party of sullen sailors.

Clayton asked no questions—he did not need to—and the following day, as the great lines of a British battle-ship grew out of the distant horizon, he half determined to demand that he and Lady Alice be put aboard her, for his fears were steadily increasing that nothing but harm could result from remaining on the lowering, sullen \emph{Fuwalda}.

Toward noon they were within speaking distance of the British vessel, but when Clayton had about decided to ask the captain to put them aboard her, the obvious ridiculousness of such a request became suddenly apparent. What reason could he give the officer commanding her majesty’s ship for desiring to go back in the direction from which he had just come!

Faith, what if he told them that two insubordinate seamen had been roughly handled by their officers. They would but laugh in their sleeves and attribute his reason for wishing to leave the ship to but one thing—cowardice.

John Clayton, Lord Greystoke, did not ask to be transferred to the British man-of-war, and late in the afternoon he saw her upper works fade below the far horizon, but not before he learned that which confirmed his greatest fears, and caused him to curse the false pride which had restrained him from seeking safety for his young wife a few short hours before, when safety was within reach—a safety which was now gone forever.

It was mid-afternoon that brought the little old sailor, who had been felled by the captain a few days before, to where Clayton and his wife stood by the ship’s side watching the ever diminishing outlines of the great battle-ship. The old fellow was polishing brasses, and as he came edging along until close to Clayton he said, in an undertone:

“\kern1pt’Ell’s to pay, sir, on this ’ere craft, an’ mark my word for it, sir. ’Ell’s to pay.”

“What do you mean, my good fellow?” asked Clayton.

“Wy, hasn’t ye seen wats goin’ on? Hasn’t ye ’eard that devil’s spawn of a capting an’ ’is mates knockin the bloomin’ lights outen ’arf the crew?

“Two busted ’eads yeste’day, an’ three today. Black Michael’s as good as new agin an’ ’e’s not the bully to stand fer it, not ’e; an’ mark my word for it, sir.”

“You mean, my man, that the crew contemplates mutiny?” asked Clayton.

“Mutiny!” exclaimed the old fellow. “Mutiny! They means murder, sir, an mark my word for it, sir.”

“When?”

“Hit’s comin’, sir; hit’s comin’ but I’m not a-sayin’ wen, an’ I’ve said too damned much now, but ye was a good sort t’other day an’ I thought it no more’n right to warn ye. But keep a still tongue in yer ’ead an’ when ye hear shootin’ git below an’ stay there.

“That’s all, only keep a still tongue in yer ’ead, or they’ll put a pill between yer ribs, an’ mark my word for it, sir,” and the old fellow went on with his polishing, which carried him away from where the Claytons were standing.

“Deuced cheerful outlook, Alice,” said Clayton.

“You should warn the captain at once, John. Possibly the trouble may yet be averted,” she said.

“I suppose I should, but yet from purely selfish motives I am almost prompted to ‘keep a still tongue in my ’ead.’ Whatever they do now they will spare us in recognition of my stand for this fellow Black Michael, but should they find that I had betrayed them there would be no mercy shown us, Alice.”

“You have but one duty, John, and that lies in the interest of vested authority. If you do not warn the captain you are as much a party to whatever follows as though you had helped to plot and carry it out with your own head and hands.”

“You do not understand, dear,” replied Clayton. “It is of you I am thinking—there lies my first duty. The captain has brought this condition upon himself, so why then should I risk subjecting my wife to unthinkable horrors in probably futile attempt to save him from his own brutal folly? You have no conception, dear, of what would follow were this pack of cutthroats to gain control of the \emph{Fuwalda}.”

“Duty is duty, my husband, and no amount of sophistries may change it. I would be a poor wife for an English lord were I to be responsible for his shirking a plain duty. I realize the danger which must follow, but I can face it with you—face it much more bravely than I could face the dishonor of always knowing that you might have averted a tragedy had you not neglected your duty.”

“Have it as you will then, Alice,” he answered, smiling. “Maybe we are borrowing trouble. While I do not like the looks of things on board this ship, they may not be so bad after all, for it is possible that the ‘Ancient Mariner’ was but voicing the desires of his wicked old heart rather than speaking of real facts.

“Mutiny on the high sea may have been common a hundred years ago, but in this good year 1888 it is the least likely of happenings.

“But there goes the captain to his cabin now. If I am going to warn him I might as well get the beastly job over for I have little stomach to talk with the brute at all.”

So saying he strolled carelessly in the direction of the companionway through which the captain had passed, and a moment later was knocking at his door.

“Come in,” growled the deep tones of that surly officer.

And when Clayton had entered, and closed the door behind him:

“Well?”

“I have come to report the gist of a conversation I heard today, because I feel that, while there may be nothing to it, it is as well that you be forearmed. In short, the men contemplate mutiny and murder.”

“It’s a lie!” roared the captain. “And if you have been interfering again with the discipline of this ship, or meddling in affairs that don’t concern you you can take the consequences, and be damned. I don’t care whether you are an English lord or not. I’m captain of this here ship, and from now on you keep your meddling nose out of my business.”

As he reached this peroration, the captain had worked himself up to such a frenzy of rage that he was fairly purple of face, and shrieked the last words at the top of his voice; emphasizing his remarks by a loud thumping of the table with one huge fist, shaking the other in Clayton’s face.

Greystoke never turned a hair, but stood eyeing the excited man with level gaze.

“Captain Billings,” he drawled finally, “if you will pardon my candor, I might remark that you are something of an ass, don’t you know.”

Whereupon he turned and left the cabin with the same indifferent ease that was habitual with him, and which was more surely calculated to raise the ire of a man of Billings’s class than a torrent of invective.

So, whereas the captain might easily have been brought to regret his hasty speech had Clayton attempted to conciliate him, his temper was now irrevocably set in the mold in which Clayton had left it, and the last chance of their working together for their common good and preservation of life was gone.

“Well, Alice,” said Clayton, as he rejoined his wife, “if I had saved my breath I should likewise have saved myself a bit of a calling. The fellow proved most ungrateful. Fairly jumped at me like a mad dog.

“He and his blasted old ship may go hang, for aught I care; and until we are safe off the thing I shall spend my energies in looking after our own welfare. And I rather fancy the first step to that end should be to go to our cabin and look over my revolvers. I am sorry now that we packed the larger guns and the ammunition with the stuff below.”

They found their quarters in a bad state of disorder. Clothing from their open boxes and bags strewed the little apartment, and even their beds had been torn to pieces.

“Evidently someone was more anxious about our belongings than we,” said Clayton. “By jove, I wonder what the bounder was after. Let’s have a look around, Alice, and see what’s missing.”

A thorough search revealed the fact that nothing had been taken but Clayton’s two revolvers and the small supply of ammunition he had saved out for them.

“Those are the very things I most wish they had left us,” said Clayton, “and the fact that they wished for them and them alone is the most sinister circumstance of all that have transpired to endanger us since we set foot on this miserable hulk.”

“What are we to do, John?” asked his wife. “I shall not urge you to go again to the captain for I cannot see you affronted further. Possibly our best chance for salvation lies in maintaining a neutral position.

“If the officers are able to prevent a mutiny, we have nothing to fear, while if the mutineers are victorious our one slim hope lies in not having attempted to thwart or antagonize them.”

“Right you are, Alice. We’ll keep in the middle of the road.”

As they fell to in an effort to straighten up their cabin, Clayton and his wife simultaneously noticed the corner of a piece of paper protruding from beneath the door of their quarters. As Clayton stooped to reach for it he was amazed to see it move further into the room, and then he realized that it was being pushed inward by someone from without.

Quickly and silently he stepped toward the door, but, as he reached for the knob to throw it open, his wife’s hand fell upon his wrist.

“No, John,” she whispered. “They do not wish to be seen, and so we cannot afford to see them. Do not forget that we are keeping the middle of the road.”

Clayton smiled and dropped his hand to his side. Thus they stood watching the little bit of white paper until it finally remained at rest upon the floor just inside the door.

Then Clayton stooped and picked it up. It was a bit of grimy, white paper roughly folded into a ragged square. Opening it they found a crude message printed in uncouth letters, with many evidences of an unaccustomed task.

Translated, it was a warning to the Claytons to refrain from reporting the loss of the revolvers, or from repeating what the old sailor had told them—to refrain on pain of death.

“I rather imagine we’ll be good,” said Clayton with a rueful smile. “About all we can do is to sit tight and wait for whatever may come.”

\clearpage